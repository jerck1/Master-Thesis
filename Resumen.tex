\newpage
\textbf{\LARGE Resumen}
\addcontentsline{toc}{chapter}{\numberline{}Resumen}\\\\
\textit{Staphylococcus aureus} es un pat\'{o}geno de importancia cl\'{i}nica que ha recibido gran atenci\'{o}n p\'{u}blica debido al desarrollo de resistencia a diferentes antibi\'{o}ticos tradicionales dirigidos a inhibir la s\'{i}ntesis de la pared de p\'{e}ptidoglicando que recubre la bacteria. Como alternativa de tratamiento se est\'{a} estudiando el mecanismo de acci\'{o}n de los p\'{e}ptidos antimicrobianos dirigidos a comprometer la integridad de la membrana lip\'{i}dica de la bacteria. Los p\'{e}ptidos antimicrobianos atacan a la membrana plasm\'{a}tica de la bacteria formando poros, los cuales disipan el potencial electroqu\'{i}mico requerido para mantener viva la bacteria. Es importante estudiar las propiedades mec\'{a}nicas de la membrana plasm\'{a}tica ya que la bacteria puede modular estas propiedades a trav\'{e}s de cambios en su composici\'{o}n lip\'{i}dica, lo cual puede resultar en incrementos en resistencia a p\'{e}ptidos antimicrobiales. Uno de los compuestos que se ha estudiado en \textit{Staphylococcus aureus} es  el carotenoide estafiloxantina, el cual aumenta  su concentraci\'{o}n en la membrana plasm\'{a}tica cuando la bacteria est\'{a} sometida a estr\'{e}s. Algunos estudios han sugerido que el papel de este carotenoide es aumentar la rigidez de la membrana plasm\'{a}tica de la bacteria. Sin embargo, a\'{u}n no est\'{a} clara la funci\'{o}n que cumple en la membrana asociada a sus propiedades biof\'{i}sicas locales, propiedades tales como la orientaci\'{o}n y las interacciones con l\'{i}pidos cercanos. Uno de los m\'{e}todos que puede contribuir al estudio del papel local de la estafiloxantina en la membrana es el de las simulaciones por din\'{a}mica molecular. En estas simulaciones se simula la trayectoria de los \'{a}tomos que conforman cada mol\'{e}cula de un fragmento peque\~no (128 l\'{i}pidos y sus mol\'{e}culas de agua asociadas) de la membrana plasm\'{a}tica.
Debido al vac\'{i}o existente en el conocimiento de estafiloxantina en \textit{Staphylococcus aureus} es objeto del presente trabajo estudiar el rol mec\'{a}nico de la estafiloxantina al insertarse en dos membranas modelo de \textit{Staphylococcus aureus}: una que contiene DMPG y otra que contiene DPPG. El rol mec\'{a}nico de la membrana se estudiar\'{a} mediante sumulaciones por din\'{a}mica molecular en las cuales se usar\'{a}n los campos de fuerza de AMBER y de CHARMM. En estos campos se optimizar\'{a}n los par\'{a}metros de los \'{a}ngulos dihedros de los diferentes grupos moleculares de estafiloxantina para tratar de reflejar el comportamiento f\'{i}sico de la mol\'{e}cula. De las simulaciones se pueden obtener propiedades biof\'{i}sicas como el par\'{a}metro de orden del deuterio, la orientaci\'{o}n de la estafiloxantina, el \'{a}rea por l\'{i}pido, entre otras. Estas propiedades pueden ser utilizadas para predecir cambios en las propiedades mec\'{a}nicas de la membrana en presencia de esta mol\'{e}cula.\\[2.0cm]
\textbf{\LARGE Abstract}\\\\
fadf\\[2.0cm]
\textbf{\small Keywords: Staphyloxanthin, \textit{Staphyloccocus aureus}, molecular dynamics, membrane biophysics}\\